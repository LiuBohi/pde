\documentclass[11pt]{article}
\usepackage{amsthm}
\usepackage{amsfonts}
\usepackage{xcolor}


\newtheorem{definition}{Definition}
\newtheorem{example}{Example}

    \title{Lecture 1: Norm and Hilbert Space}
    \author{Liu Bo}
    \date{June 25, 2020}

\begin{document}
\maketitle

\section{Norm}
\begin{definition}[Vector Space]
A \textbf{vector space} $V$ over a field $k$ is a set of vectors which come with addition ($+: V\times V \to V$) and scalar multiplication $(\cdot: k\times V\to V)$ along with some classic axioms: commutativity, associativity, identity, and inverse of addition, identity of multiplication, and distributivity.
\end{definition}

\begin{definition}[Norm]
Given a vector space $\textbf{X}$ over a subfield $F$ of the complex numbers $\mathbb{C}$, a norm is a real-values function $p$: $\textbf{X}$  $\to$ $\mathbb{R}$ with the following properties, where $|s|$ denotes the usual absolute value of a scalr s:
\begin{enumerate}
\item Positive Definite: $\| v\| \geq 0$ and $\| v\| = 0 \iff v = 0$.
\item Homogeneity: $\| \lambda v\| = |\lambda| \| v\|$ for all $v\in V$ and $\lambda \in \mathbb{R}$.
\item Triangle Inequality: $\| x+y\| \leq \| x\| + \| y\|$.
\end{enumerate}

\end{definition}

\begin{example}{Absolute-value norm}

$\| x\| = |x|$.
\end{example}

\begin{example}{Euclidean norm}

$\| \textbf{x} \|_2 =\sqrt{\textbf{x} \cdot \textbf{x}}= \sqrt{x_1^2 + \cdots + x_n^2}$
\end{example}

\begin{example}{Finite-dimensional complex normed spaces}

On an n-dimensional exomplex space $\mathbb{C}^n$, the most common is 
$\| \textbf{z}\| = \sqrt{|z_1|^2 + \cdots + |z_n|^2} = \sqrt{z_1\overline{z_1} +\cdots + z_n\overline{z_n}}$

In inner product form, this is $\left\Vert \textbf{x} \right\Vert = \sqrt{\textbf{x}^H\textbf{x}}$
\end{example}

\begin{example}{Manhattan norm}

$\left\Vert \textbf{x} \right\Vert_1 = \sum_{i=1}^n \left\vert x_i\right\vert$.
\end{example}

\begin{example}{p-norm}

Let $p\geq 1$ be a real number. The p-norm (also called $l_p$-norm) of a vector $\textbf{x}$ is 
$\left\Vert \textbf{x} \right\Vert = (\sum_{i=1}^n |x_i|^p)^{1/p}$

As p approaches $\infty$ the p-norm approaches the infinity norm:

$\left\Vert \textbf{x} \right\Vert = \max_{i}|x_i|$
\end{example}

\begin{example}{Infinite dimensions}

The generalization of the above norms to an infinite number of components leads to $l^p$ and $\textbf{L}^p$ spaces, with norms

$\left\Vert \textbf{x} \right\Vert = (\sum_{i \in \mathbb{N}} |x_i|^p)^{1/p}$ and $\left\Vert f\right\Vert_{p, \textbf{X}} = (\int_{\textbf{X}}|f(x)|^p dx)^{1/p}$
\end{example}

\section{Hilbert Space}

\begin{definition}{Inner product}

Inner product is a map

$\langle\cdot,\cdot\rangle: \textbf{V} \times \textbf{V} \to \textbf{F}$

that satisfies the following three properties for all vectors $x, t, z \in \textbf{V}$ and all scalars $a, b \in \textbf{F}$.
\begin{enumerate}
\item Conjugate symmetry: $\langle x,y\rangle=\overline{\langle y,x\rangle}$
\item Linearity in the first argument: $\langle ax+by,z\rangle=a\langle x,z\rangle+b\langle y,z\rangle$.
\item Psotive definite: if x is not zero, then $\langle x,x\rangle > 0$ 
\end{enumerate}

\end{definition}

\begin{definition}{Inner product vector space}

An \textbf{inner product vector space} is a vector space \textbf{V} over the field \textbf{F} together with an inner product.
\end{definition}

\begin{definition}{Cauchy sequency}

A sequence $x_1, x_2, x_3, \cdots$ in a metric space $(\textbf{X}, d)$ is called Cauchy if for every positive real number $r \geq 0$ there is a positive interger \textbf{N} such that for all positive integers $m, n\geq \textbf{N}$,

$d(x_m, x_n)\leq r$.
\end{definition}

\begin{definition}{Complete space}

A metric space $(\textbf{X}, d)$ is complete if every Cauchy sequence of points in \textbf{X} has a limit that is also in \textbf{X}.
\end{definition}

\begin{definition}{Hilbert Space}

A \textit{Hilbert} space \textit{H} is a real or complex inner product space that is also a complete metrix space with respect to the distance function induced by the inner product.

\end{definition}
\color{red}
good examples are needed here.
\begin{example}{Lebesgue spaces}

\end{example}

\begin{example}{Examples}

\end{example}

\end{document}